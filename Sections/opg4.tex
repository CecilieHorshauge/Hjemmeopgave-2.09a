\begin{tcolorbox}[title=Opgave 4,
    colback=blue!1!white,
    colframe=black,
    colbacktitle=blue!25!white,
    coltitle=red!25!black,
    fonttitle=\bfseries,
    subtitle style={boxrule=0.4pt,
    colback=blue!7!white} ]
    \tcbsubtitle{4a}
    Vi ved at linjen \(l\) går igennem punkterne \(A\) og \(B\).\\
    For at bestemme en parameterfremstilling for linjen \(l\) bestemmer jeg vektoren der går fra \(A\) til \(B\) og stedvektoren til \(A\).\\
    \[\begin{pmatrix} x\\y\\z \end{pmatrix}=\overrightarrow{A}+t\cdot \overrightarrow{AB}\]
    \(\overrightarrow{A}\) angiver placeringen i det tredimensionelle kartiesiske koordanatsystem. Mens \(\overrightarrow{AB}\) angiver retningen og \(t\) sørger for at vi rammer alle punkter.
    \tcbsubtitle{4b}
        Jeg bestemmer er retningsvektor for linjen \(l\):
        \[\overrightarrow{r}=\begin{pmatrix} a \\ b \\ c \end{pmatrix}\]
        Vi kender et fast punkt \(C=(x_0;\, y_0;\, z_0)=(3;\,4;\,8)\), der ligger i planen \(\pi\).\\
        Vi indsætter i planens ligning og ganger ud:
        \[a(x-x_0)+b(y-y_0)+c(z-z_0)=0\]
    \tcbsubtitle{4c}
    Vi har allerede bestemt en parameterfremstilling for linjen \(l\) og planens ligning.\\
    Vi opstiller 4 ligninger med 4 ubekendte, hvor vi bruger planens ligning og parameterfremstillingen for \(l\).
\end{tcolorbox}