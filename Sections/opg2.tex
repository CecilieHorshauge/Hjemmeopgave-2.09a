\begin{tcolorbox}[title=Opgave 2,
    colback=blue!1!white,
    colframe=black,
    colbacktitle=blue!25!white,
    coltitle=red!25!black,
    fonttitle=\bfseries,
    subtitle style={boxrule=0.4pt,
    colback=blue!7!white} ]
    \tcbsubtitle{2a}
        Vi ved at linjen \(l\) går igennem punkterne \(A\) og \(D\).\\
        For at bestemme en parameterfremstilling for linjen \(l\) bestemmer jeg vektoren der går fra \(A\) til \(D\) og stedvektoren til \(A\).\\
        \[\begin{pmatrix} x\\y\\z \end{pmatrix}=\overrightarrow{A}+t\cdot \overrightarrow{AD}\]
        \(\overrightarrow{A}\) angiver placeringen i det tredimensionelle kartiesiske koordanatsystem. Mens \(\overrightarrow{AD}\) angiver retningen og \(t\) sørger for at vi rammer alle punkter.
    \tcbsubtitle{2b}
        Først bestemmes vektor \(\overrightarrow{AB}\) og \(\overrightarrow{AD}\), de udspænder da et plan \(\pi\) der går gennem \(A,\, B,\text{ og } D\).\\
        Vi krydser \(\overrightarrow{AB}\) og \(\overrightarrow{AD}\) for at bestemme en normalvektor til planen.\\
        Dernest prikker vi normalvektoren med en vilkårlig vektor i planen.
        \[\overrightarrow{n}\bullet \left(\begin{pmatrix} x \\ y \\ z \end{pmatrix}-\begin{pmatrix}A_x \\A_y\\A_z \end{pmatrix}\right)=0\]
        Vi får planens ligning.
    \tcbsubtitle{2c}
        Vi ved at distancen mellem et punkt og en linje i rummet er givet ved:
        \[\text{dist}(P,l)=\frac{\left| \overrightarrow{r} \times \overrightarrow{P_0 P}\right|}{\left|\overrightarrow{r} \right|}\]
        hvor \(\overrightarrow{r}\) er en retningsvektor for linjen og \(\overrightarrow{P_0 P}\) er vektoren fra \(P_0\) til \(P\).
\end{tcolorbox}