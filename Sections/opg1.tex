\begin{tcolorbox}[title=Opgave 1,
    colback=blue!1!white,
    colframe=black,
    colbacktitle=blue!25!white,
    coltitle=red!25!black,
    fonttitle=\bfseries,
    subtitle style={boxrule=0.4pt,
    colback=blue!7!white} ]
    \tcbsubtitle{1a}
        Ved at betragte figuren og udnytte at alle sider står vinkelret på hinanden, kan vi ræsonnere os frem til koordinaterne til \(C,\, D,\, F \text{ og } G\).
    \tcbsubtitle{1b}
        Først bestemmes vektor \(\overrightarrow{BH}\) og \(\overrightarrow{BD}\), de udspænder da et plan \(\alpha\) der går gennem \(B,\, D,\text{ og } H\).\\
        Vi krydser \(\overrightarrow{BH}\) og \(\overrightarrow{BD}\) for at bestemme en normalvektor til planen.\\
        Dernest prikker vi normalvektoren med en vilkårlig vektor i planen.
        \[\overrightarrow{n}\bullet \left(\begin{pmatrix} x \\ y \\ z \end{pmatrix}-\begin{pmatrix}B_x \\B_y\\B_z \end{pmatrix}\right)=0\]
        Vi får planens ligning.
    \tcbsubtitle{1c}
    Først bestemmer jeg en parameterfremstilling for linjen \(l\).\\
    Dernæst udnytter jeg at vi allerede har bestemt planens ligning. Vi opstiller 4 ligninger med 4 ubekendte, hvor vi bruger planens ligning og parameterfremstillingen for \(l\).
\end{tcolorbox}