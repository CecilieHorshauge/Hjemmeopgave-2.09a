\begin{tcolorbox}[title=Opgave 6,
    colback=blue!1!white,
    colframe=black,
    colbacktitle=blue!25!white,
    coltitle=red!25!black,
    fonttitle=\bfseries,
    subtitle style={boxrule=0.4pt,
    colback=blue!7!white} ]
    \tcbsubtitle{6a}
        Brug pythagoras sætning af 2 omgange.
    \tcbsubtitle{6b}
        Vi ved at linjen \(l\) går igennem punkterne \(A\) og \(C\).\\
        For at bestemme en parameterfremstilling for linjen \(l\) bestemmer jeg vektoren der går fra \(A\) til \(C\) og stedvektoren til \(A\).\\
        \[\begin{pmatrix} x\\y\\z \end{pmatrix}=\overrightarrow{A}+t\cdot \overrightarrow{AC}\]
        \(\overrightarrow{A}\) angiver placeringen i det tredimensionelle kartiesiske koordanatsystem. Mens \(\overrightarrow{AC}\) angiver retningen og \(t\) sørger for at vi rammer alle punkter.
    \tcbsubtitle{6c}
        Først bestemmer jeg en retningsvektor for linje \(l\) og \(m\).\\
        Derefter benytter jeg sammenhængen med prikproduktet af to vektorer og vinklen imellem dem:
        \[\overrightarrow{a}\bullet \overrightarrow{b}=\left|\overrightarrow{a}\right|\cdot\left|\overrightarrow{b}\right|\cdot \text{cos}(v)\]
        Jeg isolerer v.
    \tcbsubtitle{6d}
        Vi bestemmer arealet af trekanten \(ABC\) ved:
        \[A_\triangle=\frac{1}{2} \cdot \left|\left| \overrightarrow{AB} \times \overrightarrow{AC}\right|\right|\]
        Netop da det er gældende at \(\displaystyle \left|\overrightarrow{a} \times \overrightarrow{b} \right|=\left|\overrightarrow{a}\right|\cdot \left|\overrightarrow{b} \right|\cdot \text{sin}(v)\), det vi genkender som \(h\cdot g\). 
\end{tcolorbox}